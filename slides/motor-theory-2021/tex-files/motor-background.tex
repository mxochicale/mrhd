\section{Background}

\subsection{}
%%%%%%%%%%%%%%%%%%%%%%%%%%%%%%%%%%%%%%%%%%%%%%%%%%%%%%%%
{
\paper{Page M. 1999 in {\bf http://lancet.mit.edu/motors/motors2.html}}
\begin{frame}{How does a DC motor work}

	\vspace{-3mm}
   	\begin{figure}
   		\includegraphicscopyright[width=0.75\linewidth]{introduction/pdf/how-a-motor-work}{}
 	\end{figure}


%\small 

\end{frame}
}



\subsection{}
%%%%%%%%%%%%%%%%%%%%%%%%%%%%%%%%%%%%%%%%%%%%%%%%%%%%%%%%
{
\paper{Page M. 1999 in {\bf http://lancet.mit.edu/motors/motors3.html}}
\begin{frame}{Torque}

Torque, also known as momentum, is the term used 
to talk about forces that cause or change rotational motion.

\begin{columns}
\begin{column}{0.5\textwidth}  %%<--- here

	\vspace{-4mm}
   	\begin{figure}
   		\includegraphicscopyright[width=0.65\linewidth]{introduction/pdf/torque}{}
 	\end{figure}

\end{column}

\begin{column}{0.5\textwidth}
\vspace{-20mm}
\begin{equation*}
\begin{split}
T & = r \cdot \mathbf{F} \cdot  \sin(\phi)  \\
T & = r \cdot \mathbf{F}_{tan} 
\end{split}
\end{equation*}


Units of torque \\
* SI: newton-meters (N-m) \\
* English: inch-pounds (in-lb), foot-pounds (ft-lb), inch-ounces (in-oz)
 
\end{column}


\end{columns}






\end{frame}
}


\subsection{}
%%%%%%%%%%%%%%%%%%%%%%%%%%%%%%%%%%%%%%%%%%%%%%%%%%%%%%%%
{
\paper{Page M. 1999 in {\bf http://lancet.mit.edu/motors/motors3.html}}
\begin{frame}{Speed}

The rate of rotation around an axis usually expressed 
in radians or revolutions per second or per minute.

\begin{columns}
\begin{column}{0.5\textwidth}  %%<--- here

	\vspace{-3mm}
   	\begin{figure}
   		\includegraphicscopyright[width=0.6\linewidth]{introduction/pdf/speed}{}
 	\end{figure}

\end{column}

\begin{column}{0.5\textwidth}
%\vspace{-70mm}
\begin{equation*}
\begin{split}
1 revolution & = 360 deg \\
1 revolution & = 2 \pi radians \\
1 radian & = (180/ \pi) deg \\
1 deg & = (\pi / 180) radians
\end{split}
\end{equation*}

Units of speed: \\
* radians/second (rad/s) \\
* revolutions/second (rps) \\
* revolutions/minute (rpm) 
\end{column}


\end{columns}


From the angular velocity $\omega$, we can find the tangential velocity
with $v_{tan}= r \cdot \omega$ anywhere in the rotating body.

\end{frame}
}



\subsection{}
%%%%%%%%%%%%%%%%%%%%%%%%%%%%%%%%%%%%%%%%%%%%%%%%%%%%%%%%
{
\paper{Page M. 1999 in {\bf http://lancet.mit.edu/motors/motors3.html}}
\begin{frame}{Power}

When a torque $T$ (with respect to the axis of rotation) acts 
on a body that rotates with angular velocity $\omega$,
its power (rate of doing work) is the product of the torque
and angular velocity. 

\begin{equation*}
P_{rot}= T \cdot \omega
\end{equation*}

Units of power: \\
* SI: Watts (W), newton-meters per second (N-m/s) \\
* English: foot-pound per second (ft-lb/s), horsepower (hp)

\end{frame}
}


\subsection{}
%%%%%%%%%%%%%%%%%%%%%%%%%%%%%%%%%%%%%%%%%%%%%%%%%%%%%%%
{
\paper{Page M. 1999 in {\bf http://lancet.mit.edu/motors/motors3.html}}
\begin{frame}{Torque/Speed Curves}

\begin{columns}
\begin{column}{0.4\textwidth}  %%<--- here

	\vspace{-3mm}
   	\begin{figure}
   		\includegraphicscopyright[width=0.9\linewidth]{introduction/pdf/tsc}{}
 	\end{figure}

\end{column}

\begin{column}{0.6\textwidth}

There is a tradeoff between how much torque a motor delivers, 
and how fast the output shaft spins. 

* The stall torque ($T_s$) point in the curve where
torque is at its maximum but the shaft is not rotating. \\
* The no load speed ($\omega_n$) maximum output speed
of the motor (where no torque is applied to the shaft)


\end{column}

\end{columns}

\end{frame}
}



\subsection{}
%%%%%%%%%%%%%%%%%%%%%%%%%%%%%%%%%%%%%%%%%%%%%%%%%%%%%%%
{
\paper{Page M. 1999 in {\bf http://lancet.mit.edu/motors/motors3.html}}
\begin{frame}{Torque/Speed Curves}

The area under the curve is the power given by the product 
of torque and angular velocity.
	\vspace{-3mm}
   	\begin{figure}
   		\includegraphicscopyright[width=0.85\linewidth]{introduction/pdf/tsc-power}{}
 	\end{figure}
* Due to the linear relationship of torque and angular velocity,
the maximum power occurs at the point where $\omega=1/2 \omeag_n$,
$T=1/2 T_s$.

\end{frame}
}




\subsection{}
%%%%%%%%%%%%%%%%%%%%%%%%%%%%%%%%%%%%%%%%%%%%%%%%%%%%%%%
{
\paper{Page M. 1999 in {\bf http://lancet.mit.edu/motors/motors3.html}}
\begin{frame}{Power/Torque and Power/Speed Curves}

	\vspace{-2mm}
   	\begin{figure}
   		\includegraphicscopyright[width=0.6\linewidth]{introduction/pdf/tps}{}
 	\end{figure}



\end{frame}
}


