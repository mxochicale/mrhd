\section{Motor Modelling}


\subsection{}
%%%%%%%%%%%%%%%%%%%%%%%%%%%%%%%%%%%%%%%%%%%%%%%%%%%%%%%%
{
\begin{frame}{Motor Modelling}

	%{\bf Automatic Quantification of Facial Expressions 
	%with Nonlinear Analyses in HRI}
	\vspace{-3mm}
   	\begin{figure}
   		\includegraphicscopyright[width=0.9\linewidth]{motor-equations/versions/motor-model}{}
 	\end{figure}


\end{frame}
}


%\section{Lecturing \& Technical Skills}


\subsection{}
%%%%%%%%%%%%%%%%%%%%%%%%%%%%%%%%%%%%%%%%%%%%%%%%%%%%%%%%
{
\begin{frame}{Modelling electric part of the DC motor}

By Kirchhoff's Voltage Law, we have
\begin{equation}
\begin{split}
V_{in} - V_{R} - V_{L} - V_{emf} &= 0 \\
%V_m - Ri - L \frac{di}{dt} -V_b &= 0 \\
V_{in} - Ri - L \frac{di}{dt} - k_e \dot \theta  &= 0
\end{split}
\end{equation}

Asumming the circuit has a very fast response,
Eq (1) with $L \approx 0$ is presented as
\begin{equation}
\begin{split}
V_{in} - Ri - k_e \dot \theta &=0 \\
\frac{V_{in}}{R} - \frac{k_e}{R} \dot \theta &= i
\end{split}
\end{equation}



\end{frame}
}


\subsection{}
%%%%%%%%%%%%%%%%%%%%%%%%%%%%%%%%%%%%%%%%%%%%%%%%%%%%%%%%
{
\begin{frame}{Modelling mechanical part of the DC motor}

By Newton's Law (torque aka energy balance), we have
\begin{equation} 
\begin{split}
T_e - T_{\dot \theta} - T_b - T_L = 0 \\
\end{split}
\end{equation}
Where $T_e$ is the electromagnetic torque. 
$T_{ \dot \theta }$, 
is torque generated from the rotational acceleration of the rotor. 
$T_b$, is the torque due to the friction and angular velocity in the motor. 
$T_L$ is the torque of the mechanical load (external load). 

\begin{equation}
k_t i - J \ddot  \theta - b \dot \theta  - T_L = 0
\end{equation}


\end{frame}
}


\subsection{}
%%%%%%%%%%%%%%%%%%%%%%%%%%%%%%%%%%%%%%%%%%%%%%%%%%%%%%%%
{
\begin{frame}{Complete DC motor model}

Assume the motor is not connected to the load (i.e. $T_L=0$)
in Eq 4, we have
\begin{equation}
\begin{split}
J \ddot{ \theta} + b \dot \theta  &= k_t i \\
\frac{J}{k_t} \ddot{ \theta} + \frac{b}{k_t} \dot \theta &= i
\end{split}
\end{equation}


\end{frame}
}



\subsection{}
%%%%%%%%%%%%%%%%%%%%%%%%%%%%%%%%%%%%%%%%%%%%%%%%%%%%%%%%
{
\begin{frame}{Complete DC motor model}

Subtituting $i$ from (2) in (5), we then have
\begin{equation}
\begin{split}
\frac{J}{k_t} \ddot{ \theta} + \frac{b}{k_t} \dot \theta & = \frac{V_{in}}{R} - \frac{k_e}{R} \dot \theta \\
\frac{J}{k_t} \ddot{ \theta} + (\frac{b}{k_t}  + \frac{k_e}{R} ) \dot \theta & = \frac{V_{in}}{R} \\
J \ddot{ \theta} + (b + \frac{ k_t k_e }{R} ) \dot \theta & = V_{in} \frac{k_t}{R} \\
\ddot{ \theta} + (\frac{b}{J} + \frac{ k_t k_e }{JR} ) \dot \theta & = \frac{k_t}{JR} V_{in}
\end{split}
\end{equation}


%First-order differential equation in angular velocity
%\begin{equation*}
%\begin{split}
%J \dot \omega + (b + \frac{ k_t k_e }{R} ) \omega = V_m \frac{k_t}{R}
%\end{split}
%\end{equation*}


\end{frame}
}




\subsection{}
%%%%%%%%%%%%%%%%%%%%%%%%%%%%%%%%%%%%%%%%%%%%%%%%%%%%%%%%
{
\begin{frame}{DC Motor Model}

	%{\bf Automatic Quantification of Facial Expressions 
	%with Nonlinear Analyses in HRI}
	\vspace{-3mm}
   	\begin{figure}
   		\includegraphicscopyright[width=0.7\linewidth]{motor-equations/versions/motor-model}{}
 	\end{figure}

\begin{equation*}
\begin{split}
\ddot{ \theta} + (\frac{b}{J} + \frac{ k_t k_e }{JR} ) \dot \theta & = \frac{k_t}{JR} V_{in}
\end{split}
\end{equation*}



\end{frame}
}



\subsection{}
%%%%%%%%%%%%%%%%%%%%%%%%%%%%%%%%%%%%%%%%%%%%%%%%%%%%%%%%
{
\begin{frame}{Laplace transform and transfer function of the DC motor}

\begin{equation*}
\begin{split}
\ddot{ \theta} + (\frac{b}{J} + \frac{ k_t k_e }{JR} ) \dot \theta & = \frac{k_t}{JR} V_{in}
\end{split}
\end{equation*}


\begin{equation*}
%\begin{split}
\left( s^2 + (\frac{b}{J} + \frac{ k_t k_e }{JR})s \right) \Theta(s) =  \frac{k_t}{JR} V_{in}(s)
%\end{split}
\end{equation*}


\begin{equation*}
%\begin{split}
G(s)= \frac{\Theta(s)}{V_{in}(s)} = \frac{k_t}{JR} \cdot \frac{1}{  \left( s^2 + (\frac{b}{J} + \frac{ k_t k_e }{JR})s \right) }
%\end{split}
\end{equation*}

\begin{equation}
%\begin{split}
G(s)=  \frac{k_t}{JR} \cdot \frac{1}{ s  \left( s + (\frac{b}{J} + \frac{ k_t k_e }{JR}) \right) }
%\end{split}
\end{equation}



\end{frame}
}




